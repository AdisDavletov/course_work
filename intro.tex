\chapter*{Введение}
С фактом того, что скорость появления все новых данных (информации) в глобальной сети растет, спорить не приходится. Ежедневно в различных мессенджерах пользователями генерируются сообщения в больших объемах. На тысячах сайтов оставляются огромное количество комментариев с отзывами о всевозможных продуктах и услугах. И это лишь малая часть гигантского айсберга. Все это ведет к необходимости в инструментах авотматической обработки текстов. Так, у каждого человека ежедневно возникают ситуации, когда необходимо купить какой-то продукт, в чем он не разбирается, или сходить посмотреть фильм в кинотеатре, но хочется чтобы не осталось чувства разочарвания от похода. С учетом того, что количество товаров и услуг и их разнообразие может быть огромно, то, наверняка, была бы полезной система, позволяющая ранжировать все эти товары и услуги. Подобная, хорошо реализованная, система хорошо помогала бы своим пользователям.

Так, одной из задач автоматического анализа текстов является задача построения систем, позволяющих по входному тексту определить его класс (классифицировать): отнести его к одному из заранее определенных классов. В самом простом случае, текст необходимо классифицировать на два класса: например, на положительный и отрицательный классы. Как мы видим, задача построения таких систем, обладающих высокой точностью классификации, является весьма актуальной.

В рамках курсовой работы производится сравнительный анализ существующих моделей, анализ влияния гиперпараметров и подходов к обучению на результат. Ставится целью путем применения метода взвешивания признаков из \cite{nbsvm} попробовать улучшить результаты моделей.