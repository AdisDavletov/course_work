\chapter{Заключение}
В рамках курсовой работы были реализованы модели наивного байесовского классификатора и логистической регрессии для классификации отзывов о фильмах на полжительные и отицательные классы. Экспериментальным путем было выявлено, что логистическая регрессия классифицирует тексты точнее, чем наивный байесовский классификатор. С целью улучшить результаты модели логистической регрессии был применен подход к обучению (предложенный в \cite{nbsvm}), который позволил добиться цели. В результате, на N-граммах до триграммов включительно, модель выдала результат 91.928, а при включении еще и четыреграммов, результат получился равным 92.076. Установлено, что применение взвешивания помогает лучше, чем добавление очередного N-грамма. В ходе обучения, в моделях без применения взвешивания и с его применением, формируются разные опорные признаки, сильно влияющие на результат. Есть опорные признаки присутствующие как в одной, так и в другой модели.